% Options for packages loaded elsewhere
\PassOptionsToPackage{unicode}{hyperref}
\PassOptionsToPackage{hyphens}{url}
%
\documentclass[
  ignorenonframetext,
]{beamer}
\usepackage{pgfpages}
\setbeamertemplate{caption}[numbered]
\setbeamertemplate{caption label separator}{: }
\setbeamercolor{caption name}{fg=normal text.fg}
\beamertemplatenavigationsymbolsempty
% Prevent slide breaks in the middle of a paragraph
\widowpenalties 1 10000
\raggedbottom
\setbeamertemplate{part page}{
  \centering
  \begin{beamercolorbox}[sep=16pt,center]{part title}
    \usebeamerfont{part title}\insertpart\par
  \end{beamercolorbox}
}
\setbeamertemplate{section page}{
  \centering
  \begin{beamercolorbox}[sep=12pt,center]{part title}
    \usebeamerfont{section title}\insertsection\par
  \end{beamercolorbox}
}
\setbeamertemplate{subsection page}{
  \centering
  \begin{beamercolorbox}[sep=8pt,center]{part title}
    \usebeamerfont{subsection title}\insertsubsection\par
  \end{beamercolorbox}
}
\AtBeginPart{
  \frame{\partpage}
}
\AtBeginSection{
  \ifbibliography
  \else
    \frame{\sectionpage}
  \fi
}
\AtBeginSubsection{
  \frame{\subsectionpage}
}
\usepackage{amsmath,amssymb}
\usepackage{lmodern}
\usepackage{iftex}
\ifPDFTeX
  \usepackage[T1]{fontenc}
  \usepackage[utf8]{inputenc}
  \usepackage{textcomp} % provide euro and other symbols
\else % if luatex or xetex
  \usepackage{unicode-math}
  \defaultfontfeatures{Scale=MatchLowercase}
  \defaultfontfeatures[\rmfamily]{Ligatures=TeX,Scale=1}
\fi
\usetheme[]{Ilmenau}
% Use upquote if available, for straight quotes in verbatim environments
\IfFileExists{upquote.sty}{\usepackage{upquote}}{}
\IfFileExists{microtype.sty}{% use microtype if available
  \usepackage[]{microtype}
  \UseMicrotypeSet[protrusion]{basicmath} % disable protrusion for tt fonts
}{}
\makeatletter
\@ifundefined{KOMAClassName}{% if non-KOMA class
  \IfFileExists{parskip.sty}{%
    \usepackage{parskip}
  }{% else
    \setlength{\parindent}{0pt}
    \setlength{\parskip}{6pt plus 2pt minus 1pt}}
}{% if KOMA class
  \KOMAoptions{parskip=half}}
\makeatother
\usepackage{xcolor}
\newif\ifbibliography
\usepackage{color}
\usepackage{fancyvrb}
\newcommand{\VerbBar}{|}
\newcommand{\VERB}{\Verb[commandchars=\\\{\}]}
\DefineVerbatimEnvironment{Highlighting}{Verbatim}{commandchars=\\\{\}}
% Add ',fontsize=\small' for more characters per line
\usepackage{framed}
\definecolor{shadecolor}{RGB}{248,248,248}
\newenvironment{Shaded}{\begin{snugshade}}{\end{snugshade}}
\newcommand{\AlertTok}[1]{\textcolor[rgb]{0.94,0.16,0.16}{#1}}
\newcommand{\AnnotationTok}[1]{\textcolor[rgb]{0.56,0.35,0.01}{\textbf{\textit{#1}}}}
\newcommand{\AttributeTok}[1]{\textcolor[rgb]{0.77,0.63,0.00}{#1}}
\newcommand{\BaseNTok}[1]{\textcolor[rgb]{0.00,0.00,0.81}{#1}}
\newcommand{\BuiltInTok}[1]{#1}
\newcommand{\CharTok}[1]{\textcolor[rgb]{0.31,0.60,0.02}{#1}}
\newcommand{\CommentTok}[1]{\textcolor[rgb]{0.56,0.35,0.01}{\textit{#1}}}
\newcommand{\CommentVarTok}[1]{\textcolor[rgb]{0.56,0.35,0.01}{\textbf{\textit{#1}}}}
\newcommand{\ConstantTok}[1]{\textcolor[rgb]{0.00,0.00,0.00}{#1}}
\newcommand{\ControlFlowTok}[1]{\textcolor[rgb]{0.13,0.29,0.53}{\textbf{#1}}}
\newcommand{\DataTypeTok}[1]{\textcolor[rgb]{0.13,0.29,0.53}{#1}}
\newcommand{\DecValTok}[1]{\textcolor[rgb]{0.00,0.00,0.81}{#1}}
\newcommand{\DocumentationTok}[1]{\textcolor[rgb]{0.56,0.35,0.01}{\textbf{\textit{#1}}}}
\newcommand{\ErrorTok}[1]{\textcolor[rgb]{0.64,0.00,0.00}{\textbf{#1}}}
\newcommand{\ExtensionTok}[1]{#1}
\newcommand{\FloatTok}[1]{\textcolor[rgb]{0.00,0.00,0.81}{#1}}
\newcommand{\FunctionTok}[1]{\textcolor[rgb]{0.00,0.00,0.00}{#1}}
\newcommand{\ImportTok}[1]{#1}
\newcommand{\InformationTok}[1]{\textcolor[rgb]{0.56,0.35,0.01}{\textbf{\textit{#1}}}}
\newcommand{\KeywordTok}[1]{\textcolor[rgb]{0.13,0.29,0.53}{\textbf{#1}}}
\newcommand{\NormalTok}[1]{#1}
\newcommand{\OperatorTok}[1]{\textcolor[rgb]{0.81,0.36,0.00}{\textbf{#1}}}
\newcommand{\OtherTok}[1]{\textcolor[rgb]{0.56,0.35,0.01}{#1}}
\newcommand{\PreprocessorTok}[1]{\textcolor[rgb]{0.56,0.35,0.01}{\textit{#1}}}
\newcommand{\RegionMarkerTok}[1]{#1}
\newcommand{\SpecialCharTok}[1]{\textcolor[rgb]{0.00,0.00,0.00}{#1}}
\newcommand{\SpecialStringTok}[1]{\textcolor[rgb]{0.31,0.60,0.02}{#1}}
\newcommand{\StringTok}[1]{\textcolor[rgb]{0.31,0.60,0.02}{#1}}
\newcommand{\VariableTok}[1]{\textcolor[rgb]{0.00,0.00,0.00}{#1}}
\newcommand{\VerbatimStringTok}[1]{\textcolor[rgb]{0.31,0.60,0.02}{#1}}
\newcommand{\WarningTok}[1]{\textcolor[rgb]{0.56,0.35,0.01}{\textbf{\textit{#1}}}}
\setlength{\emergencystretch}{3em} % prevent overfull lines
\providecommand{\tightlist}{%
  \setlength{\itemsep}{0pt}\setlength{\parskip}{0pt}}
\setcounter{secnumdepth}{-\maxdimen} % remove section numbering
\setbeamertemplate{navigation symbols}{}
\setbeamertemplate{footline}[page number]
\usepackage{amsmath}
\ifLuaTeX
  \usepackage{selnolig}  % disable illegal ligatures
\fi
\IfFileExists{bookmark.sty}{\usepackage{bookmark}}{\usepackage{hyperref}}
\IfFileExists{xurl.sty}{\usepackage{xurl}}{} % add URL line breaks if available
\urlstyle{same} % disable monospaced font for URLs
\hypersetup{
  pdftitle={Multivariate Analysis Lecture 5: Normal and Multivariate Normal},
  hidelinks,
  pdfcreator={LaTeX via pandoc}}

\title{Multivariate Analysis Lecture 5: Normal and Multivariate Normal}
\author{Zhaoxia Yu\\
Professor, Department of Statistics}
\date{2023-04-18}

\begin{document}
\frame{\titlepage}

\hypertarget{the-big-picture}{%
\section{The Big Picture}\label{the-big-picture}}

\begin{frame}{The Big Picture: Univariate vs Multivariate}
\protect\hypertarget{the-big-picture-univariate-vs-multivariate}{}
\begin{itemize}
\tightlist
\item
  \textcolor{red}{Review}: A random sample, denoted by
  \(X_1, \cdots, X_n\), from a (univariate) normal distribution
  \(N(\mu, \sigma^2)\)

  \begin{itemize}
  \tightlist
  \item
    What are the distributions of \(\bar X, s^2\)? What useful
    statistics can be constructed?
  \end{itemize}
\item
  \textcolor{red}{New material}: A random sample, denoted by
  \(\mathbf X_1, \cdots, \mathbf X_n\), from a multivariate normal
  distribution \(N(\boldsymbol \mu, \boldsymbol \Sigma)\)

  \begin{itemize}
  \tightlist
  \item
    What are the distributions of \(\bar{\mathbf X}, \mathbf S\)? What
    useful statistics can be constructed?
  \end{itemize}
\end{itemize}
\end{frame}

\begin{frame}{The Big Picture: Univariate}
\protect\hypertarget{the-big-picture-univariate}{}
\begin{itemize}
\tightlist
\item
  A random sample, denoted by \(X_1, \cdots, X_n\), from a (univariate)
  normal distribution \(N(\mu, \sigma^2)\)
\item
  Let \(\mathbf X_{n\times 1}=(X_1, \cdots, X_n)^T\). It is random
  vector with a multivarite normal distribution, i.e.,
  \[\mathbf X_{n\times 1}=(X_1, \cdots, X_n)^T \sim \mathbf N(\mu\mathbf 1, \sigma^2\mathbf I)\]
\end{itemize}

\begin{enumerate}
\tightlist
\item
  \(\bar X \sim N(\mu, \sigma^2/n)\)
\item
  \(\frac{(n-1)s^2}{\sigma^2} \sim \chi_{n-1}^2\)
\item
  Independence between \(\bar X\) and \(s^2\).
\item
  a t-statistic is
  \[\frac{\frac{\bar X-\mu}{\sqrt{\sigma^2/n}}}{\sqrt{\frac{(n-1)s^2/\sigma^2}{n-1}}}=\frac{\sqrt{n}(\bar X-\mu)}{s} \sim t_{n-1}\]
\end{enumerate}
\end{frame}

\begin{frame}{The Big Picture: Multivariate}
\protect\hypertarget{the-big-picture-multivariate}{}
\begin{itemize}
\tightlist
\item
  A random sample \(\mathbf X_1, \cdots, \mathbf X_n\) from a
  multivariate normal distribution
  \(\mathbf N(\boldsymbol \mu, \boldsymbol \Sigma)\).
\item
  Let \[\mathbf X_{n\times p}=\begin{pmatrix}
  \mathbf X_1^T \\ \vdots \\\mathbf X_n^T
  \end{pmatrix}\] \(\mathbf X\) follows a matrix normal distribution.
\end{itemize}

\begin{enumerate}
\item
  Sample mean vector follows a multivariate normal, i.e.,
  \(\bar{\mathbf X} \sim \mathbf N(\boldsymbol \mu, \boldsymbol \Sigma/n)\)
\item
  Sample covariance matrix \((n-1)\mathbf S\) follows a Wishart
  distribution, i.e., \((n-1)\mathbf S \sim Wishart_p (n-1, \Sigma)\)
\item
  Independence between \(\bar {\mathbf X}\) and \(S\).
\item
  Hoetelling's \(T^2\):
  \(T^2 = (\bar{\mathbf X} - \boldsymbol \mu)^T\left(\frac{\mathbf S}{n}\right)^{-1} (\bar{\mathbf X} - \boldsymbol \mu)\)
\end{enumerate}
\end{frame}

\begin{frame}{Outline}
\protect\hypertarget{outline}{}
\begin{itemize}
\tightlist
\item
  Multivariate normal distribution (MVN)
\item
  Moment generating function (MGF)

  \begin{itemize}
  \tightlist
  \item
    Apply MGF to univariate normal
  \item
    Apply MGF to multivariate normal
  \end{itemize}
\item
  Zero-Cov vs Independence
\item
  MVN: \(\bar{\mathbf X}\) and \(\mathbf S\)
\end{itemize}
\end{frame}

\hypertarget{mvn}{%
\section{MVN}\label{mvn}}

\begin{frame}{PDF of Normal of Distributions}
\protect\hypertarget{pdf-of-normal-of-distributions}{}
\begin{itemize}
\item
  Univariate normal distribution: \[
    f(x) = \frac{1}{\sqrt{2\pi\sigma^2}} e^{ -\frac{(x - \mu)^2}{2\sigma^2} }
    \]
\item
  Bivariate normal distribution: \[
    f(x, y) = \frac{1}{2\pi\sigma_1\sigma_2\sqrt{1-\rho^2}} e^{ -\frac{1}{2(1-\rho^2)} \left(\frac{(x - \mu_1)^2}{\sigma_1^2} + \frac{(y - \mu_2)^2}{\sigma_2^2} - 2\rho\frac{(x - \mu_1)(y - \mu_2)}{\sigma_1\sigma_2} \right)}
    \] The formula for a \(p\ge 3\)-dimensional multivariate normal
  distribution is much messier, so we use a compact way:
\item
  Multivariate normal distribution: \[
    f(\mathbf{x}) = \frac{1}{\sqrt{(2\pi)^p|\boldsymbol\Sigma|}} e^{ -\frac{1}{2} (\mathbf{x} - \boldsymbol\mu)^T \boldsymbol\Sigma^{-1} (\mathbf{x} - \boldsymbol\mu)}
    \]
\end{itemize}
\end{frame}

\hypertarget{mgf}{%
\section{MGF}\label{mgf}}

\begin{frame}{Tools to Characterize a Distribution}
\protect\hypertarget{tools-to-characterize-a-distribution}{}
\begin{itemize}
\tightlist
\item
  Probability density function (PDF) or probability mass function (PMF)
\item
  Cumulative distribution (CDF)
\item
  Characteristic function (CF)
\item
  Moment generating function (MGF)
\item
  \ldots{} \ldots{}
\end{itemize}
\end{frame}

\begin{frame}{Moment Generating Function (MGF)}
\protect\hypertarget{moment-generating-function-mgf}{}
\begin{itemize}
\tightlist
\item
  The moment generating function of random variable \(X\) is defined
  \[M_X(t)=\mathbb{E}[e^{tX}]\]
\item
  Like a PDF/PMF or CDF, a MGF uniquely determines/identifies a
  distribution
\item
  The definition can be extended to random vectors and random matrices

  \begin{itemize}
  \tightlist
  \item
    Consider a random vector \(\mathbf X_{p\times 1}\). Let \(t\) be a
    \(p\times 1\) vector. \[M_{\mathbf X}= \mathbb E [t^T\mathbf X]\]
  \item
    Consider a random matrix \(\mathbf X_{n\times p}\). Let \(t\) be a
    \(n\times p\) matrix.
    \[M_{\mathbf X}= \mathbb E [trace(t^T\mathbf X)]\]
  \end{itemize}
\end{itemize}
\end{frame}

\begin{frame}{Moment Generating Function: Univariate}
\protect\hypertarget{moment-generating-function-univariate}{}
\begin{itemize}
\item
  Where does the name of MGF come from? \[
  \begin{aligned}
  M_X(t) &= \mathbb{E}[e^{tX}] = \int_{-\infty}^{\infty} e^{tx} f(x) dx\\
  &=\int_{-\infty}^{\infty} [1 + tx + \frac{(tx)^2}{2!} + \frac{(tx)^3}{3!} + \cdots] f(x) dx\\
  &= 1+ t\mathbb{E}[X] + \frac{t^2}{2!}\mathbb{E}[X^2] + \cdots
  \end{aligned}
  \]
\item
  \(M^{(k)}_X(0)=E[X^K]\), where \(M^{(k)}_X(t)\) is the \(k\)th
  derivative of \(M_X(t)\).
\end{itemize}
\end{frame}

\hypertarget{mgf-univariate-normal}{%
\subsection{MGF: Univariate Normal}\label{mgf-univariate-normal}}

\begin{frame}{MGF of Univariate Normal}
\protect\hypertarget{mgf-of-univariate-normal}{}
\begin{itemize}
\item
  Recall that the MGF of a random variable X is defined as:
  \(M_X(t) = \mathbb{E}[e^{tX}]\).
\item
  For the normal distribution with mean \(\mu\) and variance
  \(\sigma^2\), the MGF is given by: \[
  M_X(t) = e^{\mu t + \frac{1}{2}\sigma^2t^2}
  \]
\item
  The mean is \(\mathbb{E}[X] = M_X'(0) = \mu\).
\item
  The variance is \[
  \begin{aligned}
  \text{Var}(X) &= E[(X-\mu)^2] = ... =E[X^2] - (E[X])^2\\
  &= M_X''(0) - M_X'(0)^2 = \sigma^2
  \end{aligned}\]
\end{itemize}
\end{frame}

\begin{frame}{MGF of Univariate Normal: Examples}
\protect\hypertarget{mgf-of-univariate-normal-examples}{}
\begin{itemize}
\tightlist
\item
  Recall that \(M_X(t)=exp\{\mu t + \frac{1}{2}{\sigma^2}t^2\}\) for
  \(X\sim N(\mu, \sigma^2)\).
\end{itemize}

What is the distribution corresponding to each of the following MGFs?

\begin{enumerate}
\item
  \[
     M_X(t) = \exp\left(\frac{1}{2}t^2\right)
     \]
\item
  \[
     M_X(t) = \exp\left(2t + \frac{9}{2}t^2\right)
     \]
\item
  \[
     M_X(t) = \exp\left(-t + \frac{1}{8}t^2\right)
     \]
\end{enumerate}
\end{frame}

\begin{frame}{MGF of Univariate Normal: Examples (continued)}
\protect\hypertarget{mgf-of-univariate-normal-examples-continued}{}
\begin{enumerate}
\item
  Standard normal distribution, i.e., \(\mu=0, \sigma^2=1\): \[
  M_X(t) = \exp\left(\frac{1}{2}t^2\right)
  \]
\item
  Normal distribution with mean \(\mu=2\) and standard deviation
  \(\sigma=3\): \[
  M_X(t) = \exp\left(2t + \frac{9}{2}t^2\right)
  \]
\item
  Normal distribution with mean \(\mu=-1\) and standard deviation
  \(\sigma=0.5\): \[
  M_X(t) = \exp\left(-t + \frac{1}{8}t^2\right)
  \]
\end{enumerate}
\end{frame}

\begin{frame}{MGF of Univariate Normal: A Linear Function}
\protect\hypertarget{mgf-of-univariate-normal-a-linear-function}{}
\begin{itemize}
\item
  Let \(X\sim N(\mu, \sigma^2)\). We know that
  \(M_X(t) = \exp\left(\mu t + \frac{1}{2}\sigma^2 t^2\right)\)
\item
  Let \(Y = aX + b\), where \(a\) and \(b\) are constants.
\item
  We now find \(M_Y(t)\): \[
  M_Y(t) = \mathbb{E}[e^{tY}] = \mathbb{E}[e^{t(aX + b)}] = e^{bt} \mathbb{E}[e^{(at)X}]
  \]
\end{itemize}

Since \(at\) is just another constant, we can treat it as a new
variable, say \(s = at\). Then: \[
\begin{aligned}
M_Y(t) &= e^{bt} M_X(s) = e^{bt} \exp\{\mu s + \frac{1}{2}\sigma^2 s^2\} \\
&= \exp\{b t + a\mu t + \frac{1}{2}\sigma^2 a^2 t^2\}
= \exp\{(a\mu + b)t + \frac{1}{2}(a\sigma)^2 t^2\}
\end{aligned} 
\]

\begin{itemize}
\tightlist
\item
  \(M_Y(t)\) has the form of the MGF of a normal distribution:
  \(Y = aX + b\sim N(a\mu+b, a^2\sigma^2)\).
\end{itemize}
\end{frame}

\begin{frame}{MGF of Univariate Normal: Sum of Two Independent Normal}
\protect\hypertarget{mgf-of-univariate-normal-sum-of-two-independent-normal}{}
\begin{itemize}
\item
  Let \(X\) and \(Y\) be two independent and
  \(X\sim N(\mu_X, \sigma_X^2)\) and \(Y\sim N(\mu_Y, \sigma_Y^2)\).
  \[M_X(t) = \exp\{\mu_X t + \frac{1}{2}\sigma_X^2 t^2\},
   M_Y(t) = \exp\{\mu_Y t + \frac{1}{2}\sigma_Y^2 t^2\}\]
\item
  Let \(Z = X + Y\). \[
  \begin{aligned}
  M_Z(t) &\overset{X\perp Y}= M_X(t) M_Y(t) = \exp\{\mu_X t + \frac{1}{2}\sigma_X^2 t^2\} \exp\{\mu_Y t + \frac{1}{2}\sigma_Y^2 t^2\}\\
  &=\exp\{(\mu_X+\mu_Y) t + \frac{1}{2}(\sigma_X^2+\sigma_Y^2) t^2\}
  \end{aligned}
  \] Which indicates that
  \(Z\sim N(\mu_X+\mu_Y, \sigma_X^2+\sigma_Y^2)\)
\end{itemize}
\end{frame}

\begin{frame}{MGF of Univariate Normal: Sample Mean}
\protect\hypertarget{mgf-of-univariate-normal-sample-mean}{}
\begin{itemize}
\tightlist
\item
  If \(X_1, \cdots, X_n \overset{iid}\sim N(\mu, \sigma^2)\).
\item
  We have showed that \(E[\bar X]=\mu\) and \(Var[\bar X]=\sigma^2/n\).
\item
  How to prove \(\bar X\) follows a normal distribution?
\item
  A compact proof: \[
  M_{\bar{X}}(t) = \prod_{i=1}^{n} M_{X_i}(\frac{t}{n})= \left(\exp\{\mu\frac{t}{n} + \frac{1}{2}\sigma^2 \frac{t^2}{n^2}\}\right)^n
  =\exp\{\mu t + \frac{1}{2}\frac{\sigma^2}{n}t^2 \}
  \] Based on the \(M_{\bar X}(t)\),
  \(\bar X\sim N(\mu, \frac{\sigma^2}{n})\).
\end{itemize}
\end{frame}

\begin{frame}{MGF of Univariate Normal: Sample Mean}
\protect\hypertarget{mgf-of-univariate-normal-sample-mean-1}{}
\begin{itemize}
\tightlist
\item
  A proof with more details explained \[
  \begin{aligned}
  M_{\bar X}(t)&=E[e^{t\bar X}]= E[e^{\frac{t}{n} \sum_{i=1}^n X_i}]=E[e^{\frac{t}{n} X_1 + \frac{t}{n} X_2 + \cdots + \frac{t}{n} X_n}]\\
  &\overset{iid}=E[e^{\frac{t}{n} X_1}]\cdots E[e^{\frac{t}{n} X_n}]=M_{X_1}(\frac{t}{n}) \cdots M_{X_n}(\frac{t}{n})\\
  &= \exp\{\mu\frac{t}{n}+\frac{1}{2}\sigma^2 (\frac{t}{n})^2\}\cdots \exp\{\mu\frac{t}{n}+\frac{1}{2}\sigma^2 (\frac{t}{n})^2\}\\
  &= \exp\{\mu t + \frac{1}{2}\frac{\sigma^2}{n}t^2\}
  \end{aligned}
  \] Based on the \(M_{\bar X}(t)\),
  \(\bar X\sim N(\mu, \frac{\sigma^2}{n})\).
\end{itemize}
\end{frame}

\hypertarget{mgf-of-mvn}{%
\subsection{MGF of MVN}\label{mgf-of-mvn}}

\begin{frame}{MGF of Multivariate Normal}
\protect\hypertarget{mgf-of-multivariate-normal}{}
\begin{itemize}
\tightlist
\item
  The moment generating function (MGF) of a random vector
  \(\mathbf{X}_{p\times 1}\) is defined as:
  \(M_{\mathbf{X}}(\mathbf{t}) = \mathbb{E}[e^{\mathbf{t}^T\mathbf{X}}]\).
\item
  Here \(t\) is a \(p\times 1\) vector.\\
\item
  For the multivariate normal distribution with mean vector
  \(\boldsymbol{\mu}\) and covariance matrix \(\boldsymbol{\Sigma}\),
  the MGF is given by:
\end{itemize}

\[
M_{\mathbf{X}}(\mathbf{t}) = \exp\left( \boldsymbol{\mu}^T \mathbf{t} + \frac{1}{2} \mathbf{t}^T \boldsymbol{\Sigma} \mathbf{t} \right)
\]
\end{frame}

\begin{frame}{MGF of MVN: Examples}
\protect\hypertarget{mgf-of-mvn-examples}{}
\begin{enumerate}
\item
  Bivariate standard normal distribution:

  \[\boldsymbol{\mu}=\begin{pmatrix}0 \\ 0\end{pmatrix}, \boldsymbol{\Sigma}=\begin{pmatrix}1 & 0 \\ 0 & 1\end{pmatrix}, M_{\mathbf{X}}(\mathbf{t}) = \exp\left(\frac{1}{2} \mathbf{t}^T \boldsymbol{\Sigma} \mathbf{t}\right)
  \]
\item
  Bivariate normal distribution with specific mean vector and covariance
  matrix:
\end{enumerate}

\[\boldsymbol{\mu}=\begin{pmatrix}1 \\ 2\end{pmatrix},
\boldsymbol{\Sigma}=\begin{pmatrix}4 & 1 \\ 1 & 9\end{pmatrix},    M_{\mathbf{X}}(\mathbf{t}) = \exp\left(\boldsymbol{\mu}^T \mathbf{t} + \frac{1}{2} \mathbf{t}^T \boldsymbol{\Sigma} \mathbf{t}\right)
\]
\end{frame}

\begin{frame}{MGF of MVN: A Linear Combination}
\protect\hypertarget{mgf-of-mvn-a-linear-combination}{}
\begin{itemize}
\item
  Let
  \(\mathbf{X}_{p\times 1}\sim N(\boldsymbol \mu, \boldsymbol \Sigma)\).
\item
  We want to show that the linear combinations
  \(\mathbf{Y} = \mathbf{A}_{q\times p}\mathbf{X}\) also follows a
  multivariate normal distribution.
\item
  The MGF of \(\mathbf X\) is \[
  M_{\mathbf{X}}(\mathbf{t}) = E[e^{t^T\mathbf X}]=\exp\left(\boldsymbol{\mu}^T \mathbf{t} + \frac{1}{2} \mathbf{t}^T \boldsymbol{\Sigma} \mathbf{t}\right)
  \]
\end{itemize}
\end{frame}

\begin{frame}{MGF of MVN: A Linear Combination}
\protect\hypertarget{mgf-of-mvn-a-linear-combination-1}{}
\begin{itemize}
\item
  To find the distribution of \(Y\), we derive the MGF of \(Y\). \[
  \begin{aligned}
  M_{\mathbf{Y}}(\mathbf{t}) &= E[e^{t^TAX}]=M_{\mathbf{X}}(\mathbf{A}^T \mathbf{t}) = \exp\left(\boldsymbol{\mu}^T \mathbf{A}^T \mathbf{t} + \frac{1}{2} \mathbf{t}^T \mathbf{A} \boldsymbol{\Sigma} \mathbf{A}^T \mathbf{t}\right)\\
  &=\exp\left((\mathbf{A}\boldsymbol{\mu})^T \mathbf{t} + \frac{1}{2} \mathbf{t}^T (\mathbf{A}\boldsymbol{\Sigma}\mathbf{A}^T) \mathbf{t}\right),
  \end{aligned}
  \]
\item
  \(M_{\mathbf{Y}}(\mathbf{t})\) has the form of the MGF of a
  multivariate normal distribution with mean vector
  \(\mathbf{A}\boldsymbol{\mu}\) and covariance matrix
  \(\mathbf{A}\boldsymbol{\Sigma}\mathbf{A}^T\).
\item
  As a result, the linear combination
  \[\mathbf Y =A \mathbf X \sim N(A \boldsymbol \mu, \boldsymbol A \boldsymbol \Sigma A^T)\]
\end{itemize}
\end{frame}

\begin{frame}{MGF of MVN: The Sample Mean Vector}
\protect\hypertarget{mgf-of-mvn-the-sample-mean-vector}{}
\begin{itemize}
\tightlist
\item
  Let \(\mathbf X_1, \cdots \mathbf X_n\) be a random sample from
  \(N(\boldsymbol \mu, \boldsymbol \Sigma)\).
\item
  We have defined then the sample mean vector \(\bar{\mathbf{X}}\)
\item
  We have shown that

  \begin{itemize}
  \tightlist
  \item
    \(\mathbb E[\bar{\mathbf{X}}]=\boldsymbol \mu\)
  \item
    \(Cov[\bar{\mathbf{X}}]=\frac{\boldsymbol \Sigma}{n}\)
  \end{itemize}
\item
  Next, we will show that it follows a multivariate normal distribution.
\end{itemize}
\end{frame}

\begin{frame}{MGF of MVN: The Sample Mean Vector}
\protect\hypertarget{mgf-of-mvn-the-sample-mean-vector-1}{}
\begin{itemize}
\tightlist
\item
  We first calculate its MGF: \[
  \begin{aligned}
  M_{\bar{\mathbf{X}}}(\mathbf{t}) &\overset{iid}= \prod_{i=1}^n M_{\mathbf{X}_i}(\frac{1}{n}\mathbf{t}) = \left(\exp\left(\boldsymbol{\mu}^T \frac{1}{n}\mathbf{t} + \frac{1}{2} \left(\frac{1}{n}\mathbf{t}\right)^T \boldsymbol{\Sigma} \left(\frac{1}{n}\mathbf{t}\right)\right)\right)^n\\
  &= \exp\left(n\left(\boldsymbol{\mu}^T \frac{1}{n}\mathbf{t} + \frac{1}{2n^2} \mathbf{t}^T \boldsymbol{\Sigma} \mathbf{t}\right)\right) \\
  &= \exp\left(\boldsymbol{\mu}^T \mathbf{t} + \frac{1}{2} \mathbf{t}^T \frac{1}{n}\boldsymbol{\Sigma} \mathbf{t}\right)
  \end{aligned}
  \]
\item
  \(M_{\bar{\mathbf{X}}}(\mathbf{t})\) has the form of the MGF of a
  multivariate normal distribution with mean vector \(\boldsymbol{\mu}\)
  and covariance matrix \(\frac{1}{n}\boldsymbol{\Sigma}\), i.e,
  \(\bar{\mathbf{X}}\sim N(\boldsymbol \mu , \frac{\boldsymbol \Sigma}{n})\)
\end{itemize}
\end{frame}

\hypertarget{zero-covariance}{%
\section{Zero-Covariance}\label{zero-covariance}}

\begin{frame}{Independence of Normals Under Jointly Normal}
\protect\hypertarget{independence-of-normals-under-jointly-normal}{}
\begin{itemize}
\item
  In general, zero-correlation does not guarantee independence
\item
  Independence of normals under jointly normal: If the joint
  distribution of \(\mathbf X_1\) (a \(p\times 1\) random vector) and
  \(\mathbf X_2\) is jointly/multivariate normal, i.e., \[
  \begin{pmatrix}\mathbf X_1 \\ \mathbf X_2
  \end{pmatrix} \sim N(
  \begin{pmatrix}\boldsymbol \mu_1\\\boldsymbol \mu_2\end{pmatrix},
  \begin{pmatrix}\boldsymbol \Sigma_{11} & \boldsymbol \Sigma_{12}\\ \boldsymbol \Sigma_{21} & \boldsymbol \Sigma_{22}\end{pmatrix}),
  \] then
  \(\mathbf X_1 \perp \mathbf X_2 \Leftrightarrow \boldsymbol \Sigma_{12}=0\)
\item
  Proof: omitted. A result about MGF can be used to prove independence.
\end{itemize}
\end{frame}

\begin{frame}{The Joint Distribution of Two Linear Functions}
\protect\hypertarget{the-joint-distribution-of-two-linear-functions}{}
\begin{itemize}
\tightlist
\item
  Let
  \(\mathbf{X}_{p\times 1}\sim N(\boldsymbol \mu, \boldsymbol \Sigma)\).
\item
  Let \(\mathbf Y=A \mathbf X\) and \(\mathbf Z=B \mathbf X\).
\item
  What is the joint distribution of \(Y\) and \(Z\)?
\item
  Note that \[\begin{pmatrix}
  \mathbf Y\\ \mathbf Z
  \end{pmatrix}=\begin{pmatrix}A \\ B\end{pmatrix}\mathbf X
  \]
\item
  By the results in the two previous slides\textcolor{red}{???} \[
  \begin{aligned}
  \mathbf Y & \sim N(\begin{pmatrix}A \\ B\end{pmatrix}\boldsymbol \mu, \begin{pmatrix}A \\ B\end{pmatrix}\boldsymbol \Sigma \begin{pmatrix}A^T & B^T\end{pmatrix})\\
  &\sim N(\begin{pmatrix}A \boldsymbol \mu \\ B\boldsymbol \mu \end{pmatrix}, \begin{pmatrix}A \boldsymbol \Sigma A^T & A \boldsymbol \Sigma B^T\\ B \boldsymbol \Sigma A^T & B \boldsymbol \Sigma B^T \end{pmatrix}\boldsymbol)
  \end{aligned} 
  \] We have \[\mathbf Y \perp \mathbf Z \Leftrightarrow A\Sigma B^T=0\]
\end{itemize}
\end{frame}

\begin{frame}[fragile]{A Simulation Question}
\protect\hypertarget{a-simulation-question}{}
\begin{itemize}
\tightlist
\item
  Suppose we have a random sample from a normal distribution.
\item
  How to use a simulation to show that sample mean and sample variance
  are uncorrelated (in fact they are also independent)?
\end{itemize}

\begin{Shaded}
\begin{Highlighting}[]
\CommentTok{\# Proof of independence between sample mean and sample variance for univariate case}
\CommentTok{\#library(mvtnorm)}

\CommentTok{\# Generate random sample from normal distribution}
\FunctionTok{set.seed}\NormalTok{(}\DecValTok{123}\NormalTok{)}
\NormalTok{n }\OtherTok{\textless{}{-}} \DecValTok{100}
\NormalTok{mu }\OtherTok{\textless{}{-}} \DecValTok{0}
\NormalTok{sigma2 }\OtherTok{\textless{}{-}} \DecValTok{1}
\NormalTok{X }\OtherTok{\textless{}{-}} \FunctionTok{rnorm}\NormalTok{(n, mu, }\FunctionTok{sqrt}\NormalTok{(sigma2))}

\CommentTok{\# Calculate sample mean and sample variance}
\NormalTok{mean\_X }\OtherTok{\textless{}{-}} \FunctionTok{mean}\NormalTok{(X)}
\NormalTok{var\_X }\OtherTok{\textless{}{-}} \FunctionTok{var}\NormalTok{(X)}

\CommentTok{\# Verify that sample mean and sample variance are unbiased estimators}
\CommentTok{\#assertthat::assert\_that(mean\_X == mu)}
\CommentTok{\#assertthat::assert\_that(var\_X == sigma2)}

\CommentTok{\# Use Cochran\textquotesingle{}s theorem to show independence}
\NormalTok{k }\OtherTok{\textless{}{-}} \DecValTok{1} \CommentTok{\# degrees of freedom for sample mean}
\NormalTok{nu }\OtherTok{\textless{}{-}}\NormalTok{ n }\SpecialCharTok{{-}} \DecValTok{1} \CommentTok{\# degrees of freedom for sample variance}
\NormalTok{F\_stat }\OtherTok{\textless{}{-}}\NormalTok{ k }\SpecialCharTok{*}\NormalTok{ (n }\SpecialCharTok{{-}} \DecValTok{1}\NormalTok{) }\SpecialCharTok{*}\NormalTok{ var\_X }\SpecialCharTok{/}\NormalTok{ (sigma2 }\SpecialCharTok{*}\NormalTok{ nu)}
\NormalTok{p\_value }\OtherTok{\textless{}{-}} \FunctionTok{pf}\NormalTok{(F\_stat, }\AttributeTok{df1=}\NormalTok{k, }\AttributeTok{df2=}\NormalTok{nu, }\AttributeTok{lower.tail=}\ConstantTok{FALSE}\NormalTok{)}
\ControlFlowTok{if}\NormalTok{ (p\_value }\SpecialCharTok{\textgreater{}} \FloatTok{0.05}\NormalTok{) \{}
  \FunctionTok{cat}\NormalTok{(}\StringTok{"Reject null hypothesis at 5\% significance level}\SpecialCharTok{\textbackslash{}n}\StringTok{"}\NormalTok{)}
\NormalTok{\} }\ControlFlowTok{else}\NormalTok{ \{}
  \FunctionTok{cat}\NormalTok{(}\StringTok{"Fail to reject null hypothesis at 5\% significance level}\SpecialCharTok{\textbackslash{}n}\StringTok{"}\NormalTok{)}
\NormalTok{\}}
\end{Highlighting}
\end{Shaded}
\end{frame}

\hypertarget{sample-mean-and-sample-variance}{%
\subsection{Sample Mean and Sample
Variance}\label{sample-mean-and-sample-variance}}

\begin{frame}{The Independence Between Sample Mean and Sample Variance}
\protect\hypertarget{the-independence-between-sample-mean-and-sample-variance}{}
\begin{itemize}
\item
  For a random sample from a normal distribution, the sample mean and
  sample variance are independent.
\item
  Let \(\mathbf X =(X_1, X_2, ..., X_n)^T\) be a random sample from a
  normal distribution with mean \(\mu\) and variance \(\sigma^2\).
\item
  The sample mean and sample variance are defined as:

  \begin{itemize}
  \item
    Sample mean: \(\bar{X} = \frac{1}{n} \sum_{i=1}^{n} X_i\)
  \item
    Sample variance:
    \(s^2 = \frac{1}{n-1} \sum_{i=1}^{n} (X_i - \bar{X})^2\)
  \end{itemize}
\item
  We want to show that \(\bar{X}\) and \(s^2\) are independent.
\end{itemize}
\end{frame}

\begin{frame}{Proof}
\protect\hypertarget{proof}{}
\begin{itemize}
\item
  We first rewrite the sample mean:
  \[\bar{X} = \frac{1}{n} \sum_{i=1}^{n} X_i =\frac{1}{n}\mathbf 1^T \mathbf X \]
\item
  We have shown that
  \(s^2 = \frac{1}{n-1} \sum_{i=1}^{n} (X_i - \bar{X})^2=\frac{1}{n-1}X^T\mathbb CX\),
  where \(\mathbb C=\mathbf I - \frac{1}{n}\mathbf 1 \mathbf 1^T\). In
  addition, it is easy to verify that
  \(\mathbb C=\mathbb C^T, \mathbb C^2=\mathbb C\). Thus, the
  distribution of \(s^2\) can be rewritten to
  \[s^2=\frac{1}{n-1} (\mathbb C\mathbf X)^T (\mathbb C\mathbf X),\]
  which indicates that the distribution \(s^2\) is determined by the
  distribution of \(\mathbb C X\).
\item
  Clearly \(\bar {\mathbf X}\) and \(\mathbb C\) are linear combinations
  of \(\mathbf X\), which follows a multivariate normal with covariance
  \(\boldsymbol\Sigma= \mathbf I\). Thus,
  \[cov(\bar {\mathbf X}, \mathbb C) = \frac{1}{n}\mathbf 1^T \Sigma \mathbb C= \frac{1}{n}\mathbf 1^T \mathbb C=0\]
  Please verify that last step.
\end{itemize}

By Theorem on ``Independence of Normals Under Jointly Normal'', we can
conclude the \(\bar {\mathbf X}\) and \(s^2\) are independent.
\end{frame}

\hypertarget{mvn-bar-mathbf-x-and-s}{%
\section{\texorpdfstring{MVN: \(\bar {\mathbf X}\) and
\(S\)}{MVN: \textbackslash bar \{\textbackslash mathbf X\} and S}}\label{mvn-bar-mathbf-x-and-s}}

\begin{frame}{The Independence Between Sample Mean Vector and Sample
Covariance Matrix}
\protect\hypertarget{the-independence-between-sample-mean-vector-and-sample-covariance-matrix}{}
\begin{itemize}
\tightlist
\item
  How to prove that the sample mean vector and the sample covariance
  matrix are independent
\item
  Messier way: vectorize the \(n\times p\) matrix \(\mathbf X\) to a
  \((np)\times 1\) vector and then apply the condition for independent
  linear combinations under MVN
\item
  Neater way: use properties of Matrix Normal Distribution
\end{itemize}
\end{frame}

\begin{frame}{Sample Mean Vector and Sample Covariance Matrix}
\protect\hypertarget{sample-mean-vector-and-sample-covariance-matrix}{}
\begin{itemize}
\tightlist
\item
  If we have a random sample from MVN, we will show that
  \(\bar{\mathbf X}\) and \(S\) are independent
\item
  Proof outline

  \begin{enumerate}
  \tightlist
  \item
    Vectorize \(\mathbf X_{n\times p}\) to a vector
    \(\tilde {\mathbf X}_{(np)\times 1}\), which follows a MVN
  \item
    Show that the distribution of \(\bar{\mathbf X}\) is determined by a
    linear function of \(\tilde {\mathbf X}_{(np)\times 1}\)
  \item
    Show that the distribution of \(S\) is determined by a linear
    function of \(\tilde {\mathbf X}_{(np)\times 1}\)
  \item
    Find the covariance of the two linear functions
  \item
    Conclude that the two linear functions are independent, which
    indicates that the sample mean vector and the sample covariance
    matrix are independent
  \end{enumerate}
\end{itemize}
\end{frame}

\begin{frame}{Step 1a: Vectorize}
\protect\hypertarget{step-1a-vectorize}{}
\begin{itemize}
\tightlist
\item
  We vectorize \(\mathbf X_{n\times p}\) such that

  \begin{itemize}
  \tightlist
  \item
    the first \(n\) random variables are for the first feature
  \item
    \ldots{} \ldots{}
  \item
    the last \(n\) random variables are for the last feature
    \[\tilde {\mathbf X}_{(np)\times 1}=
    \begin{pmatrix}
    \mathbf X_{(1)}\\ \vdots \\ \mathbf X_{(p)} \end{pmatrix}\]
  \end{itemize}
\item
  What is the distribution of \(\mathbf X_{(1)}\)?
\end{itemize}
\end{frame}

\begin{frame}{Step 1b: The distribution of \(\mathbf x_{n\times p}\)}
\protect\hypertarget{step-1b-the-distribution-of-mathbf-x_ntimes-p}{}
\[
\begin{aligned}
\tilde {\mathbf X}_{(np)\times 1} &=
\begin{pmatrix}
\mathbf X_{(1)}\\ \vdots \\ \mathbf X_{(p)} \end{pmatrix}
\sim N(\boldsymbol \mu \otimes\mathbf 1_n, 
\boldsymbol \Sigma \otimes \mathbf I_n)\\
&\sim 
N (
\begin{pmatrix}\mu_1 \mathbf 1_n \\ \vdots \\\mu_p \mathbf 1_n\end{pmatrix},  
\begin{pmatrix}
\sigma_{11}\mathbf I_n & \cdots & \sigma_{1p}\mathbf I_n\\
\cdots & \cdots & \cdots \\
\sigma_{p1}\mathbf I_n & \cdots & \sigma_{pp}\mathbf I_n
\end{pmatrix})
\end{aligned}
\]
\end{frame}

\begin{frame}{Step 2: The sample mean vector}
\protect\hypertarget{step-2-the-sample-mean-vector}{}
\begin{itemize}
\tightlist
\item
  The sample mean vector can be written as linear functions of
  \(\tilde {\mathbf X}\):
  \[\bar{\mathbf X}=\frac{1}{n} \begin{pmatrix} \mathbf 1_n^T & \mathbf 0_n^T & \cdots &\mathbf 0_n^T \\
  \mathbf 0_n^T & \mathbf 1_n^T & \cdots &\mathbf 0_n^T \\
  \cdots & \cdots & \cdots & \cdots \\
  \mathbf 0_n^T & \mathbf 0_n^T & \cdots &\mathbf 1_n^T 
  \end{pmatrix} \tilde {\mathbf X}\]
\end{itemize}
\end{frame}

\begin{frame}{Step 3: The sample covariance matrix}
\protect\hypertarget{step-3-the-sample-covariance-matrix}{}
\begin{itemize}
\tightlist
\item
  Recall that we have shown the following result
  \[S=\frac{1}{n-1} \mathbf X^T \mathbb C \mathbf X= \frac{1}{n-1} (\mathbb C \mathbf X)^T \mathbb C \mathbf X\]
\item
  So we only need to focus on \(\mathbb C \mathbf X\), the centered
  random matrix.
\item
  The vectorized version of the centered random matrix is
  \[vec(\mathbb C \mathbf X) = 
  \begin{pmatrix}
  \mathbb C & \mathbf 0 & \cdots & \mathbf 0\\
  \mathbf 0 & \mathbb C & \cdots & \mathbf 0\\
  \cdots & \cdots & \cdots & \cdots\\
  \mathbf 0 & \mathbf 0 & \cdots & \mathbb C
  \end{pmatrix}_{(np)\times (np)} \tilde {\mathbf X}
  \]
\end{itemize}
\end{frame}

\begin{frame}{Step 4: The covariance of the two linear functions}
\protect\hypertarget{step-4-the-covariance-of-the-two-linear-functions}{}
So we have the following results - \[
\tilde {\mathbf X}_{(np)\times 1} 
\sim N(\boldsymbol \mu \otimes\mathbf 1_n, 
\boldsymbol \Sigma \otimes \mathbf I_n)
\sim 
N (
\begin{pmatrix}\mu_1 \mathbf 1_n \\ \vdots \\\mu_p \mathbf 1_n\end{pmatrix},  
\begin{pmatrix}
\sigma_{11}\mathbf I_n & \cdots & \sigma_{1p}\mathbf I_n\\
\cdots & \cdots & \cdots \\
\sigma_{p1}\mathbf I_n & \cdots & \sigma_{pp}\mathbf I_n
\end{pmatrix})
\] - The distribution of \(\bar{\mathbf X}\) and \(S\) depend on
\[\bar{\mathbf X}=\frac{1}{n} \begin{pmatrix} 
\mathbf 1_n^T & \mathbf 0_n^T & \cdots &\mathbf 0_n^T \\
\mathbf 0_n^T & \mathbf 1_n^T & \cdots &\mathbf 0_n^T \\
\cdots & \cdots & \cdots & \cdots \\
\mathbf 0_n^T & \mathbf 0_n^T & \cdots &\mathbf 1_n^T 
\end{pmatrix} \tilde {\mathbf X},  vec(\mathbb C \mathbf X) = 
\begin{pmatrix}
\mathbb C & \mathbf 0 & \cdots & \mathbf 0\\
\mathbf 0 & \mathbb C & \cdots & \mathbf 0\\
\cdots & \cdots & \cdots & \cdots\\
\mathbf 0 & \mathbf 0 & \cdots & \mathbb C
\end{pmatrix} \tilde {\mathbf X}
\]
\end{frame}

\begin{frame}{Step 4: The covariance of the two linear functions}
\protect\hypertarget{step-4-the-covariance-of-the-two-linear-functions-1}{}
\begin{itemize}
\tightlist
\item
  Let \(\tilde {\boldsymbol \Sigma}\) denote the covariance matrix of
  \(\tilde {\mathbf x}\)
\item
  Let \(\mathbb A\) denote the matrix such that
  \(\bar{\mathbf X}=\mathbb A \tilde {\mathbf x}\)
\item
  Let \(\mathbb B\) denote the matrix such that
  \(vec(\mathbb C\mathbf X)=\mathbb B \tilde {\mathbf X}\)
\item
  It can be verified that
  \(\mathbb A \tilde {\boldsymbol \Sigma} \mathbb B^T=\mathbf 0\).
\end{itemize}
\end{frame}

\begin{frame}{Step 5: The independnece of the two linear functions}
\protect\hypertarget{step-5-the-independnece-of-the-two-linear-functions}{}
\begin{itemize}
\tightlist
\item
  Both \(\bar{\mathbf X}\) and \(vec(\mathbb C\mathbf X)\) are linear
  function of the same MVN-distributed random vector
  \(\tilde {\mathbf X}\)
\item
  Their covariance matrix is zero, which indicates that they are
  independent by Theorem on ``Independence of Normals Under Jointly
  Normal''.
\item
  The sample covariance matrix only depends on the centered
  data,\(vec(\mathbb C\mathbf X)\) (the vector form) up to a constant
\item
  Therefor, if we have a random sample from a \textcolor{red}{MVN}, the
  sample mean vector and the sample covariance matrix are independent
\item
  The proof is lengthy. It can be more compact if we introduce matrix
  normal distribution.
\end{itemize}
\end{frame}

\end{document}
